\documentclass[]{academic_template}

% single-column: \documentclass[]{academic_template}
% twocolumn: \documentclass[twocolumn]{academic_template}

\usepackage[toc,page,header]{appendix}

%%%%%%%%%%%%%%%%%%%%%%%%%%%%%%%%%%%%

\usepackage{amssymb}
\usepackage{amsmath}

\usepackage{makecell}
\usepackage{algorithm}      % 算法浮动环境
\usepackage{algpseudocode}  % 算法伪代码语法

\usepackage{graphicx}

\usepackage{listings}
\lstset{
  basicstyle=\ttfamily\small,
  breaklines=true,       % 自动换行
  columns=fullflexible,  % 更好换行
  keepspaces=true,       % 保留空格
  showstringspaces=false
}

\usepackage{url}
\usepackage{booktabs}
\usepackage{wrapfig}
\newcommand{\tablestyle}[2]{\setlength{\tabcolsep}{#1}\renewcommand{\arraystretch}{#2}\centering\footnotesize}

\usepackage{colortbl}
\usepackage{xcolor}

\usepackage{caption}
\usepackage{subcaption}

\newcommand{\fix}{\marginpar{FIX}}
\newcommand{\new}{\marginpar{NEW}}
\def\eg{\emph{e.g.}}
\def\ie{\emph{i.e.}}



%%%%%%%%%%%%%%%%%%%%%
%! 建议在此处修改模板的配色和logo大小等参数

% \setthemecolor{8B008B} % 主题颜色,默认深蓝,这个是深紫色
\setlogoheight{14mm}   % logo size
% \sectionlineon %增加章节分割线
\setlogospacing{5mm}
\setsjtublue % 设置logo为上交蓝,默认为上交红

\settitlerulethickness{3pt}  % title line size

% 对abstract的调整
\abstractboxon %显示边框
\setabstractframecolor{gray} %灰色边框
\setabstractbgcolor{gray!10} %浅灰色背景
\setlogotolineshift{5mm} % logo与标题行的距离

% 设置标题上下横线的粗度(默认 0.5pt)
\settoprulethickness{2.5pt}  
\setbottomrulethickness{1.5pt} 



%%%%%%%%%%%%%%%%%%%


\title{\LaTeX Template for VisionXLab@RethinkLab: A Comprehensive Template for Academic Writing}

\author[1,*]{First Author}
\author[1,2]{Second Author}
\author[2,3]{Third Author}
\author[1, \dagger]{Fourth Author with a looooooooooooooooooooong name}

% 论文单位
\affiliation[1]{Your University}
\affiliation[2]{Your Institute or Company}
\affiliation[3]{A Company with a loooooooooooooooooooooooooooooooooooooooooong name}

\contribution[*]{Equal contribution}
\contribution[\dagger]{Corresponding Author}

\abstract{
This is a placeholder abstract for your academic paper template. The abstract should provide a concise summary of your research, including the motivation, main contributions, methodology, and key results. Typically, an abstract ranges from 150 to 250 words and should be self-contained, allowing readers to understand the essence of your work without reading the entire paper.

In this template, we demonstrate various LaTeX elements commonly used in academic papers, including figures, tables, algorithms, equations, and citations. This template is designed to be highly extensible and suitable for various academic disciplines. You can easily customize the content, styling, and structure to fit your specific needs.
}

\date{\today}
% Uncomment below to add correspondence email
% \correspondence{Author Name at \email{your.email@example.com}}

% Additional info fields
\checkdata[Project Page]{\url{https://your-project-page.github.io/}}
\checkdata[Code]{\url{https://github.com/your-repo}}
\checkdata[Hugging Face]{\url{https://huggingface.co/your-repo}}
% 放宽浮动体放置限制
\renewcommand{\topfraction}{0.9}      % 顶部能放浮动体的最大比例
\renewcommand{\bottomfraction}{0.8}   % 底部能放浮动体的最大比例
\renewcommand{\textfraction}{0.07}    % 一页中文字必须占据的最小比例
\renewcommand{\floatpagefraction}{0.85} % 浮动页(全是表格的页)的最小占用比例
\begin{document}
\maketitle

% 如果需要目录,取消下面的注释
% \newpage
% \tableofcontents
% \newpage

\section{Introduction}
\label{sec:intro}

% 这是一个包含多种元素的示例章节

% 示例图片 - 双栏宽度
\begin{figure}[!tb]
\centering
\includegraphics[width=0.9\linewidth]{figures/content/example_figure.pdf}
\caption{\textbf{Example figure with full-width layout.} This demonstrates how to include a figure that spans both columns in a two-column layout. Replace this placeholder with your actual figure. The figure should be in PDF format for vector graphics to ensure high quality and scalability.}
\label{fig:example}
\end{figure}

% 段落 1: 背景和动机
Recent advances in [your research field] have significantly transformed [the application domain]~\cite{example2023paper}. Despite impressive progress in [specific aspect], several fundamental challenges remain unaddressed. In particular, [describe the main problem] requires a comprehensive understanding of [key factors] and careful consideration of [important constraints].

% 段落 2: 现有方法的局限性
Prior work has explored various approaches to tackle this problem~\cite{smith2022method, johnson2023approach}. However, existing methods exhibit clear limitations when applied to [specific scenarios]:

\begin{itemize}
\item \textbf{First limitation}: [Describe the first major limitation]. For example, traditional approaches typically [describe behavior] while [describe the actual requirement].

\item \textbf{Second limitation}: [Describe the second limitation]. This results in [negative consequences] and prevents [desired outcomes].

\item \textbf{Third limitation}: [Describe the third limitation]. The lack of [missing component] leads to [specific problems] in [particular contexts].
\end{itemize}

% 段落 3: 我们的解决方案
In this work, we address these challenges by proposing \textbf{YourMethodName}, a novel framework that [high-level description of your approach]. Our key insight is that [describe your key insight or observation]. Specifically, we introduce [list your main technical contributions] to achieve [your goals].

% 段落 4: 主要贡献
The main contributions of this paper are summarized as follows:

\begin{itemize}
\item We propose [first contribution], which enables [benefit or capability].

\item We introduce [second contribution], providing [advantage] and addressing [specific problem].

\item We develop [third contribution], yielding [improved performance or new capability].

\item We conduct comprehensive experiments on [datasets/benchmarks], demonstrating that our method achieves [performance improvements] compared to state-of-the-art approaches.
\end{itemize}

% 示例数学公式
As illustrated in Equation~\ref{eq:example}, our approach can be formalized as:
\begin{equation}
\label{eq:example}
\mathcal{L}_{\text{total}} = \mathcal{L}_{\text{task}} + \lambda \mathcal{L}_{\text{reg}}
\end{equation}
where $\mathcal{L}_{\text{task}}$ represents the task-specific loss, $\mathcal{L}_{\text{reg}}$ is the regularization term, and $\lambda$ controls the trade-off between the two objectives.

% 段落 5: 论文组织结构
The remainder of this paper is organized as follows. Section~\ref{sec:related} reviews related work in [relevant areas]. Section~\ref{sec:method} presents our proposed method in detail. Section~\ref{sec:experiments} describes our experimental setup and reports comprehensive results. Finally, Section~\ref{sec:conclusion} concludes the paper and discusses future directions.

\section{Related Work}
\label{sec:related}

% 本章节展示如何组织相关工作,包含多个子主题

% 子主题 1
\subsection{[First Research Area]}
\label{subsec:area1}

Early work in [first research area] primarily focused on [early approaches]~\cite{author2020early}. These methods typically relied on [key techniques or assumptions] and achieved [certain results]. For instance, [Author et al.]~\cite{author2021specific} proposed [method name], which [description]. However, this approach was limited by [limitations].

More recent advances have explored [newer directions]~\cite{recent2023work}. [Describe recent developments and their contributions]. Despite these improvements, [remaining challenges or gaps] continue to pose significant obstacles.

% 子主题 2
\subsection{[Second Research Area]}
\label{subsec:area2}

% 示例表格 - 比较不同方法
\begin{table}[!tb]
\centering
\caption{Comparison of different approaches in [research area]. This table summarizes the key characteristics and limitations of existing methods.}
\label{tab:comparison}
\tablestyle{4pt}{1.2}
\begin{tabular}{lccc}
\toprule
\textbf{Method} & \textbf{Feature A} & \textbf{Feature B} & \textbf{Limitation} \\
\midrule
Method 1~\cite{example2023paper} & \checkmark & \texttimes & Limited scalability \\
Method 2~\cite{smith2022method} & \checkmark & \checkmark & High complexity \\
Method 3~\cite{johnson2023approach} & \texttimes & \checkmark & Poor generalization \\
\textbf{Ours} & \checkmark & \checkmark & -- \\
\bottomrule
\end{tabular}
\end{table}

Another line of research has investigated [second research area]~\cite{baseline2022paper}. [Describe this research direction and its relevance to your work]. As shown in Table~\ref{tab:comparison}, existing methods differ in [key aspects]. While [Method X]~\cite{methodx2023} achieves [certain advantages], it suffers from [specific drawbacks]. Similarly, [Method Y]~\cite{methody2022} addresses [problem A] but fails to handle [problem B].

% 子主题 3
\subsection{[Third Research Area]}
\label{subsec:area3}

The integration of [concept A] with [concept B] has gained increasing attention in recent years~\cite{integration2023}. [Describe this integration and why it matters]. Several notable works have made progress in this direction:

\begin{itemize}
\item \textbf{[Work 1]}~\cite{work1}: [Brief description of contribution and limitation]
\item \textbf{[Work 2]}~\cite{work2}: [Brief description of contribution and limitation]
\item \textbf{[Work 3]}~\cite{work3}: [Brief description of contribution and limitation]
\end{itemize}

% 子主题 4: 与本文工作的关系
\subsection{Relation to Our Work}
\label{subsec:relation}

Our work is most closely related to [specific area or methods]~\cite{close2023work}. However, we differ in several key aspects. First, unlike [previous approach], our method [describe key difference 1]. Second, we introduce [novel component], which enables [capability not available before]. Third, our framework [describe architectural or methodological difference].

Furthermore, while prior work has primarily focused on [narrow scope], we provide a more comprehensive solution that addresses [broader challenges]. The experimental results in Section~\ref{sec:experiments} demonstrate that our approach achieves significant improvements over these existing methods.

\section{Method}
\label{sec:method}

% 本章节展示技术方法,包含算法、公式、架构图等

In this section, we present the details of our proposed method. We first provide an overview of the framework architecture in Section~\ref{subsec:overview}, followed by detailed descriptions of the key components in Sections~\ref{subsec:component1}--\ref{subsec:component3}.

% 方法概述
\subsection{Overview}
\label{subsec:overview}

% 示例架构图
\begin{figure}[!tb]
\centering
\includegraphics[width=0.6\linewidth]{figures/content/architecture.pdf}
\caption{\textbf{Overview of our proposed framework.} The architecture consists of three main components: (a) [Component 1] for [purpose], (b) [Component 2] for [purpose], and (c) [Component 3] for [purpose]. The data flows from left to right, with [describe the processing pipeline].}
\label{fig:architecture}
\end{figure}

As illustrated in Figure~\ref{fig:architecture}, our framework takes [input description] and produces [output description] through [number] main stages. The overall objective can be formulated as:

\begin{equation}
\label{eq:objective}
\min_{\theta} \mathbb{E}_{(x,y) \sim \mathcal{D}} \left[ \ell(f_\theta(x), y) \right] + \Omega(\theta)
\end{equation}

where $\theta$ denotes the model parameters, $\mathcal{D}$ is the data distribution, $f_\theta$ is our model, $\ell$ is the loss function, and $\Omega(\theta)$ represents the regularization term.

% 第一个组件
\subsection{[Component 1 Name]}
\label{subsec:component1}

The first component of our framework is designed to [purpose and motivation]. Given an input $\mathbf{x} \in \mathbb{R}^{d}$, we first apply [transformation]:

\begin{equation}
\label{eq:component1}
\mathbf{h} = \text{Encoder}(\mathbf{x}) = \mathbf{W}_e \mathbf{x} + \mathbf{b}_e
\end{equation}

where $\mathbf{W}_e \in \mathbb{R}^{d_h \times d}$ and $\mathbf{b}_e \in \mathbb{R}^{d_h}$ are learnable parameters. The resulting representation $\mathbf{h}$ captures [what it captures].

To enhance [specific aspect], we introduce [novel mechanism]:

\begin{equation}
\label{eq:attention}
\alpha_{ij} = \frac{\exp(\text{score}(\mathbf{h}_i, \mathbf{h}_j))}{\sum_{k=1}^{n} \exp(\text{score}(\mathbf{h}_i, \mathbf{h}_k))}
\end{equation}

where $\text{score}(\cdot, \cdot)$ measures the compatibility between different representations.

% 第二个组件
\subsection{[Component 2 Name]}
\label{subsec:component2}

% 示例算法伪代码
\begin{algorithm}[!tb]
\caption{[Algorithm Name]}
\label{alg:main}
\begin{algorithmic}[1]
\Require Input data $\mathbf{X} = \{\mathbf{x}_1, \ldots, \mathbf{x}_n\}$, hyperparameters $\lambda, \eta$
\Ensure Output predictions $\mathbf{Y} = \{\mathbf{y}_1, \ldots, \mathbf{y}_n\}$
\State Initialize parameters $\theta \leftarrow \theta_0$
\For{$t = 1$ to $T$}
    \State Sample mini-batch $\mathcal{B} \subset \{1, \ldots, n\}$
    \For{$i \in \mathcal{B}$}
        \State Compute forward pass: $\hat{\mathbf{y}}_i = f_\theta(\mathbf{x}_i)$
        \State Compute loss: $\mathcal{L}_i = \ell(\hat{\mathbf{y}}_i, \mathbf{y}_i)$
    \EndFor
    \State Compute gradient: $\nabla_\theta \mathcal{L} = \frac{1}{|\mathcal{B}|} \sum_{i \in \mathcal{B}} \nabla_\theta \mathcal{L}_i$
    \State Update parameters: $\theta \leftarrow \theta - \eta \nabla_\theta \mathcal{L}$
\EndFor
\State \Return $\mathbf{Y} = \{f_\theta(\mathbf{x}_i)\}_{i=1}^{n}$
\end{algorithmic}
\end{algorithm}

Building on the representations from the previous component, we design [Component 2] to [purpose]. The complete procedure is summarized in Algorithm~\ref{alg:main}. The key idea is to [describe key idea] by iteratively [describe iteration process].

At each step $t$, we update [what gets updated] according to:

\begin{equation}
\label{eq:update}
\mathbf{z}^{(t+1)} = \mathbf{z}^{(t)} + \alpha \cdot \nabla_{\mathbf{z}} \mathcal{F}(\mathbf{z}^{(t)}, \mathbf{h})
\end{equation}

where $\mathcal{F}$ is an energy function and $\alpha$ is the step size.

% 第三个组件
\subsection{[Component 3 Name]}
\label{subsec:component3}

% 子图示例
\begin{figure}[!tb]
\centering
\makebox[\linewidth][c]{%
\begin{subfigure}{0.28\linewidth}
    \centering
    \includegraphics[width=\linewidth]{figures/content/sub_figure_a.pdf}
    \caption{Visualization of [aspect A]}
    \label{fig:sub_a}
\end{subfigure}
\hspace{0.08\linewidth}% 固定间距,避免贴边
\begin{subfigure}{0.28\linewidth}
    \centering
    \includegraphics[width=\linewidth]{figures/content/sub_figure_b.pdf}
    \caption{Visualization of [aspect B]}
    \label{fig:sub_b}
\end{subfigure}
}% 使整体块居中,左右边距均匀
\caption{\textbf{Detailed visualization of [component].} (a) shows [description of subplot a], while (b) illustrates [description of subplot b]. These visualizations demonstrate [what they demonstrate].}
\label{fig:visualization}
\end{figure}

The final component integrates outputs from the previous stages to produce the final prediction. As shown in Figure~\ref{fig:visualization}, [describe what the figure shows]. Mathematically, this can be expressed as:

\begin{equation}
\label{eq:final}
\hat{\mathbf{y}} = g(\mathbf{z}, \mathbf{h}) = \text{Decoder}(\mathbf{z} \oplus \mathbf{h})
\end{equation}

where $\oplus$ denotes the fusion operation and $g$ is the final prediction function.

% 损失函数
\subsection{Training Objective}
\label{subsec:training}

To train our model end-to-end, we define a multi-task loss function that combines [different objectives]:

\begin{equation}
\label{eq:total_loss}
\mathcal{L}_{\text{total}} = \mathcal{L}_{\text{pred}} + \lambda_1 \mathcal{L}_{\text{aux}} + \lambda_2 \mathcal{L}_{\text{reg}}
\end{equation}

where:
\begin{itemize}
\item $\mathcal{L}_{\text{pred}}$ is the main prediction loss, typically defined as:
\begin{equation}
\mathcal{L}_{\text{pred}} = \frac{1}{N} \sum_{i=1}^{N} \|\hat{\mathbf{y}}_i - \mathbf{y}_i\|^2
\end{equation}

\item $\mathcal{L}_{\text{aux}}$ is an auxiliary loss that encourages [specific property]

\item $\mathcal{L}_{\text{reg}}$ is a regularization term to prevent overfitting

\item $\lambda_1, \lambda_2$ are hyperparameters that balance the different objectives
\end{itemize}

% 实现细节
\subsection{Implementation Details}
\label{subsec:implementation}

We implement our framework using [framework name, \eg PyTorch]. The model is trained for [number] epochs with batch size [size] using the Adam optimizer~\cite{kingma2014adam} with learning rate [lr]. We employ [specific techniques like dropout, batch normalization, etc.] to improve training stability and generalization. All experiments are conducted on [hardware specifications].

\section{Experiments}
\label{sec:experiments}

% 本章节展示实验设置和结果,包含多种类型的表格和图表

In this section, we evaluate our proposed method through comprehensive experiments. We first describe the experimental setup in Section~\ref{subsec:setup}, followed by quantitative results in Section~\ref{subsec:quantitative} and qualitative analysis in Section~\ref{subsec:qualitative}. Finally, we present ablation studies in Section~\ref{subsec:ablation}.

% 实验设置
\subsection{Experimental Setup}
\label{subsec:setup}

\paragraph{Datasets}
We evaluate our method on [number] widely-used benchmarks:
\begin{itemize}
\item \textbf{Dataset 1}~\cite{dataset1}: Contains [number] samples with [characteristics]. We follow the standard split of [train/val/test].
\item \textbf{Dataset 2}~\cite{dataset2}: A challenging dataset featuring [characteristics]. We use [number] samples for training and [number] for testing.
\item \textbf{Dataset 3}~\cite{dataset3}: Designed for [specific task], consisting of [description].
\end{itemize}

\paragraph{Evaluation Metrics}
Following prior work~\cite{baseline2022paper}, we use the following metrics:
\begin{itemize}
\item \textbf{Metric 1}: Measures [what it measures]
\item \textbf{Metric 2}: Evaluates [what it evaluates]
\item \textbf{Metric 3}: Quantifies [what it quantifies]
\end{itemize}

\paragraph{Baselines}
We compare our method against several state-of-the-art approaches:
\begin{itemize}
\item \textbf{Method A}~\cite{methoda}: [Brief description]
\item \textbf{Method B}~\cite{methodb}: [Brief description]
\item \textbf{Method C}~\cite{methodc}: [Brief description]
\item \textbf{Method D}~\cite{methodd}: [Brief description]
\end{itemize}

\paragraph{Implementation Details}
We implement our method using PyTorch and train on 4 NVIDIA A100 GPUs. The model is optimized using AdamW with a learning rate of $2 \times 10^{-4}$ and weight decay of $0.01$. We train for 100 epochs with batch size 32. Data augmentation includes random cropping, horizontal flipping, and color jittering.

% 定量结果
\subsection{Quantitative Results}
\label{subsec:quantitative}

% 主要结果表格
\begin{table*}[!tb]
\centering
\caption{Quantitative comparison on three benchmarks. Best results are in \textbf{bold}, and second best are \underline{underlined}. $\uparrow$ indicates higher is better, $\downarrow$ indicates lower is better.}
\label{tab:main_results}
\tablestyle{6pt}{1.3}
\begin{tabular}{l|ccc|ccc|ccc}
\toprule
\multirow{2}{*}{\textbf{Method}} & \multicolumn{3}{c|}{\textbf{Dataset 1}} & \multicolumn{3}{c|}{\textbf{Dataset 2}} & \multicolumn{3}{c}{\textbf{Dataset 3}} \\
& Metric1$\uparrow$ & Metric2$\uparrow$ & Metric3$\downarrow$ & Metric1$\uparrow$ & Metric2$\uparrow$ & Metric3$\downarrow$ & Metric1$\uparrow$ & Metric2$\uparrow$ & Metric3$\downarrow$ \\
\midrule
Method A~\cite{methoda} & 72.3 & 68.5 & 0.245 & 65.8 & 62.1 & 0.312 & 70.2 & 66.8 & 0.278 \\
Method B~\cite{methodb} & 75.6 & 71.2 & 0.223 & 68.4 & 65.3 & 0.289 & 73.5 & 69.7 & 0.251 \\
Method C~\cite{methodc} & 78.1 & 73.8 & 0.208 & 71.2 & 68.6 & 0.267 & 76.3 & 72.4 & 0.234 \\
Method D~\cite{methodd} & \underline{80.5} & \underline{76.3} & \underline{0.195} & \underline{74.6} & \underline{71.2} & \underline{0.248} & \underline{79.1} & \underline{75.6} & \underline{0.219} \\
\midrule
\textbf{Ours} & \textbf{83.7} & \textbf{79.8} & \textbf{0.178} & \textbf{77.9} & \textbf{74.5} & \textbf{0.231} & \textbf{82.4} & \textbf{78.9} & \textbf{0.203} \\
\bottomrule
\end{tabular}
\end{table*}

Table~\ref{tab:main_results} presents the quantitative comparison on three benchmarks. Our method consistently outperforms all baselines across all datasets and metrics. Specifically, compared to the best baseline (Method D), our approach achieves:
\begin{itemize}
\item \textbf{Dataset 1}: +3.2\% on Metric1, +3.5\% on Metric2, and -8.7\% on Metric3
\item \textbf{Dataset 2}: +3.3\% on Metric1, +3.3\% on Metric2, and -6.9\% on Metric3
\item \textbf{Dataset 3}: +3.3\% on Metric1, +3.3\% on Metric2, and -7.3\% on Metric3
\end{itemize}

These improvements demonstrate the effectiveness of our proposed components, particularly [Component X] which addresses [specific limitation of baselines].

% 不同设置下的结果
\begin{table}[!tb]
\centering
\caption{Performance under different settings on Dataset 1. Our method maintains robust performance across various conditions.}
\label{tab:different_settings}
\tablestyle{4pt}{1.2}
\begin{tabular}{lcccc}
\toprule
\textbf{Setting} & \textbf{Baseline} & \textbf{Method D} & \textbf{Ours} & \textbf{Gain} \\
\midrule
Standard & 75.6 & 80.5 & \textbf{83.7} & +3.2 \\
Low-resource & 68.2 & 72.8 & \textbf{76.4} & +3.6 \\
High-noise & 62.5 & 67.3 & \textbf{71.1} & +3.8 \\
Cross-domain & 58.9 & 63.7 & \textbf{68.2} & +4.5 \\
\bottomrule
\end{tabular}
\end{table}

% 定性结果
\subsection{Qualitative Results}
\label{subsec:qualitative}

% 示例比较图
\begin{figure}[!tb]
\centering
\includegraphics[width=0.6\linewidth]{figures/content/qualitative_comparison.pdf}
\caption{\textbf{Qualitative comparison with baseline methods.} From left to right: Input, Ground Truth, Method B, Method D, and Ours. Our method produces results that are more [describe advantages], particularly in [challenging scenarios]. Red boxes highlight regions where our method shows significant improvements.}
\label{fig:qualitative}
\end{figure}

Figure~\ref{fig:qualitative} shows qualitative comparisons between our method and baselines. As can be observed, our approach generates more [quality attribute] results, especially in challenging cases involving [difficult scenarios]. While Method D performs reasonably well in [certain aspects], it struggles with [specific challenges]. In contrast, our method successfully handles these cases by [explain why your method works].

% 消融实验
\subsection{Ablation Study}
\label{subsec:ablation}

% 消融实验表格
\begin{table}[!tb]
\centering
\caption{Ablation study on Dataset 1. Each row removes one component from the full model to analyze its contribution.}
\label{tab:ablation}
\tablestyle{4pt}{1.2}
\begin{tabular}{lccc}
\toprule
\textbf{Variant} & \textbf{Metric1$\uparrow$} & \textbf{Metric2$\uparrow$} & \textbf{Metric3$\downarrow$} \\
\midrule
Full Model & \textbf{83.7} & \textbf{79.8} & \textbf{0.178} \\
\midrule
w/o Component 1 & 79.2 & 75.4 & 0.201 \\
w/o Component 2 & 80.8 & 77.1 & 0.189 \\
w/o Component 3 & 81.5 & 78.2 & 0.184 \\
w/o Loss Term 1 & 82.1 & 78.6 & 0.182 \\
w/o Loss Term 2 & 82.9 & 79.2 & 0.180 \\
\bottomrule
\end{tabular}
\end{table}

To validate the effectiveness of each component, we conduct ablation studies by removing individual components from the full model. As shown in Table~\ref{tab:ablation}, each component contributes to the overall performance:

\begin{itemize}
\item \textbf{Component 1} has the largest impact (-4.5\% on Metric1), confirming its importance for [its role].
\item \textbf{Component 2} contributes -2.9\% on Metric1, demonstrating its value in [its purpose].
\item \textbf{Component 3} provides -2.2\% improvement, showing that [its benefit].
\item Both loss terms contribute to performance, with Loss Term 1 being more critical.
\end{itemize}

% 超参数分析
\subsection{Hyperparameter Analysis}
\label{subsec:hyperparameter}

\begin{figure}[!tb]
\centering
\includegraphics[width=0.6\linewidth]{figures/content/hyperparameter_analysis.pdf}
\caption{\textbf{Impact of key hyperparameters.} We analyze the effect of (left) $\lambda_1$ and (right) $\lambda_2$ on performance. The model is relatively robust to hyperparameter choices within reasonable ranges.}
\label{fig:hyperparameter}
\end{figure}

We analyze the sensitivity of our method to key hyperparameters. As shown in Figure~\ref{fig:hyperparameter}, performance is relatively stable across a wide range of $\lambda_1$ and $\lambda_2$ values, indicating that our method is robust and does not require extensive hyperparameter tuning. The optimal performance is achieved when $\lambda_1 = 0.1$ and $\lambda_2 = 0.5$.

\section{Conclusion}
\label{sec:conclusion}

% 本章节总结全文并讨论未来工作

In this paper, we have presented \textbf{[YourMethodName]}, a novel framework for [main task/problem]. Our key contributions include:

\begin{itemize}
\item A [describe first contribution] that enables [benefit]
\item A [describe second contribution] that addresses [limitation]
\item A [describe third contribution] that improves [aspect]
\end{itemize}

Through comprehensive experiments on [number] benchmark datasets, we have demonstrated that our method significantly outperforms existing state-of-the-art approaches. Specifically, our approach achieves [X\%] improvement on [metric] while maintaining [desirable property]. The ablation studies confirm that each component of our framework contributes meaningfully to the overall performance.

% 局限性
\subsection{Limitations}
\label{subsec:limitations}

Despite the promising results, our method has several limitations that warrant future investigation:

\begin{itemize}
\item \textbf{Computational Cost}: The current implementation requires [computational resources], which may limit its applicability to [certain scenarios]. Future work could explore more efficient architectures or approximation techniques to reduce computational requirements.

\item \textbf{Data Requirements}: Our method benefits from large-scale training data. In low-resource settings, performance may degrade. Investigating few-shot learning or transfer learning approaches could help address this limitation.

\item \textbf{Specific Scenarios}: While our method performs well on the tested benchmarks, it may struggle with [specific challenging cases]. Further research is needed to enhance robustness in these scenarios.
\end{itemize}

% 未来工作
\subsection{Future Directions}
\label{subsec:future}

Several promising directions for future research include:

\begin{itemize}
\item \textbf{Extension to Other Domains}: While we have focused on [current domain], the proposed framework is general and could potentially be applied to [other domains such as X, Y, Z]. Exploring these applications could reveal new insights and broaden the impact of our work.

\item \textbf{Integration with [Technique X]}: Recent advances in [related technique] suggest potential synergies with our approach. Investigating how to effectively combine these methods could lead to further performance improvements.

\item \textbf{Real-world Deployment}: Transitioning from research prototypes to practical systems requires addressing additional challenges such as [deployment challenges]. Future work should focus on [specific aspects] to enable real-world applications.

\item \textbf{Theoretical Analysis}: While our empirical results are strong, a deeper theoretical understanding of [why the method works] would be valuable. Developing formal guarantees or convergence proofs could strengthen the foundation of our approach.
\end{itemize}

% 总结
In conclusion, this work represents a significant step forward in [research area] by introducing [main innovation]. We hope that our contributions will inspire future research and facilitate progress toward [ultimate goal]. The code and models will be made publicly available to support reproducibility and further development by the research community.

% 致谢(如果需要可以添加)
% \section*{Acknowledgments}
% We thank [people/organizations] for [their contributions]. This work was supported by [funding sources].


\clearpage

\bibliographystyle{plainnat}
\bibliography{references}

% 如果需要附录,取消下面三行的注释,反之加上注释
\clearpage
\beginappendix

\subsection{Comparison between previous method}

\$CONTENTS\$

\end{document}
