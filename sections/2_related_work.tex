\section{Related Work}
\label{sec:related}

% 本章节展示如何组织相关工作,包含多个子主题

% 子主题 1
\subsection{[First Research Area]}
\label{subsec:area1}

Early work in [first research area] primarily focused on [early approaches]~\cite{author2020early}. These methods typically relied on [key techniques or assumptions] and achieved [certain results]. For instance, [Author et al.]~\cite{author2021specific} proposed [method name], which [description]. However, this approach was limited by [limitations].

More recent advances have explored [newer directions]~\cite{recent2023work}. [Describe recent developments and their contributions]. Despite these improvements, [remaining challenges or gaps] continue to pose significant obstacles.

% 子主题 2
\subsection{[Second Research Area]}
\label{subsec:area2}

% 示例表格 - 比较不同方法
\begin{table}[!tb]
\centering
\caption{Comparison of different approaches in [research area]. This table summarizes the key characteristics and limitations of existing methods.}
\label{tab:comparison}
\tablestyle{4pt}{1.2}
\begin{tabular}{lccc}
\toprule
\textbf{Method} & \textbf{Feature A} & \textbf{Feature B} & \textbf{Limitation} \\
\midrule
Method 1~\cite{example2023paper} & \checkmark & \texttimes & Limited scalability \\
Method 2~\cite{smith2022method} & \checkmark & \checkmark & High complexity \\
Method 3~\cite{johnson2023approach} & \texttimes & \checkmark & Poor generalization \\
\textbf{Ours} & \checkmark & \checkmark & -- \\
\bottomrule
\end{tabular}
\end{table}

Another line of research has investigated [second research area]~\cite{baseline2022paper}. [Describe this research direction and its relevance to your work]. As shown in Table~\ref{tab:comparison}, existing methods differ in [key aspects]. While [Method X]~\cite{methodx2023} achieves [certain advantages], it suffers from [specific drawbacks]. Similarly, [Method Y]~\cite{methody2022} addresses [problem A] but fails to handle [problem B].

% 子主题 3
\subsection{[Third Research Area]}
\label{subsec:area3}

The integration of [concept A] with [concept B] has gained increasing attention in recent years~\cite{integration2023}. [Describe this integration and why it matters]. Several notable works have made progress in this direction:

\begin{itemize}
\item \textbf{[Work 1]}~\cite{work1}: [Brief description of contribution and limitation]
\item \textbf{[Work 2]}~\cite{work2}: [Brief description of contribution and limitation]
\item \textbf{[Work 3]}~\cite{work3}: [Brief description of contribution and limitation]
\end{itemize}

% 子主题 4: 与本文工作的关系
\subsection{Relation to Our Work}
\label{subsec:relation}

Our work is most closely related to [specific area or methods]~\cite{close2023work}. However, we differ in several key aspects. First, unlike [previous approach], our method [describe key difference 1]. Second, we introduce [novel component], which enables [capability not available before]. Third, our framework [describe architectural or methodological difference].

Furthermore, while prior work has primarily focused on [narrow scope], we provide a more comprehensive solution that addresses [broader challenges]. The experimental results in Section~\ref{sec:experiments} demonstrate that our approach achieves significant improvements over these existing methods.
